%====================================================================
\documentclass[11pt,a4paper]{article}
%====================================================================
\usepackage[utf8]{inputenc}
\usepackage{amsmath,empheq}
\usepackage{amsfonts}
\usepackage{amssymb}
\usepackage{fourier}
\usepackage[left=2.5cm,right=2.5cm,top=2.5cm,bottom=2.5cm]{geometry}
\author{Ali SAAD}
\title{Formulation of Conservation Equations for the Shrinkage-Induced Surface Deformation}
%====================================================================
%%%   PACKAGES   %%%

%\usepackage{xcolor}
\usepackage[dvipsnames]{xcolor} 
\definecolor{halfgray}{gray}{0.55}
\definecolor{webgreen}{rgb}{0,.5,0}
\definecolor{webbrown}{rgb}{.6,0,0}
\definecolor{Maroon}{cmyk}{0, 0.87, 0.68, 0.32}
\definecolor{RoyalBlue}{cmyk}{1, 0.50, 0, 0}
\definecolor{Black}{cmyk}{0, 0, 0, 0}


\usepackage[pdftex,hyperfootnotes=false,pdfpagelabels]{hyperref}
\hypersetup{%
    %draft,	% = no hyperlinking at all (useful in b/w printouts)
    colorlinks=true, linktocpage=true, pdfstartpage=3, pdfstartview=FitV,%
    % uncomment the following line if you want to have black links (e.g., for printing)
    %colorlinks=false, linktocpage=false, pdfborder={0 0 0}, pdfstartpage=3, pdfstartview=FitV,% 
    breaklinks=true, pdfpagemode=UseNone, pageanchor=true, pdfpagemode=UseOutlines,%
    plainpages=false, bookmarksnumbered, bookmarksopen=true, bookmarksopenlevel=1,%
    hypertexnames=true, pdfhighlight=/O,%nesting=true,%frenchlinks,%
    urlcolor=webbrown, linkcolor=RoyalBlue, citecolor=webgreen, %pagecolor=RoyalBlue,%
    %urlcolor=Black, linkcolor=Black, citecolor=Black, %pagecolor=Black,%
} 

\usepackage{cancel}
\usepackage{import}

\usepackage{tabularx} % better tables
	\setlength{\extrarowheight}{3pt} % increase table row height
\newcommand{\tableheadline}[1]{\multicolumn{1}{c}{\spacedlowsmallcaps{#1}}}
\newcommand{\myfloatalign}{\centering} % to be used with each float for alignment
\usepackage{caption}
\captionsetup{format=hang,font=small}
%\usepackage{subfig}
\usepackage{float}
\usepackage{subcaption}

%\usepackage{minitoc}
%====================================================================
%%%   COMMANDS   %%%
%\renewcommand{matrix}[1]{#1 \rangle} 
\newcommand{\avg}[1]{\left\langle #1 \right\rangle}
\newcommand{\brac}[1]{ \left(  #1  \right) } 
\newcommand{\mtrx}[1]{{\underline{\underline{#1}}} }
\newcommand\mat[1]{\ensuremath{\underline{#1}}} 

\newcommand\nabvec{\ensuremath{\vec{\nabla}}}
\newcommand\nabmat{\ensuremath{\mat{\nabla}}}
\newcommand\temp[1]{\ensuremath{\frac{\partial}{\partial t} \brac{#1} }}
\newcommand\tempn[1]{\ensuremath{\frac{\partial #1}{\partial t} }}  
 
\newcommand\vit{\ensuremath{\avg{\vec{v}^l}}}  
\newcommand\strainrate{\ensuremath{\avg{{\dot{\mat{\varepsilon}}}^l}}}

\newcommand\wavg{\ensuremath{\avg{w}}}
\newcommand\wl{\ensuremath{\avg{w}^l}}
\newcommand\ws{\ensuremath{\avg{w}^s}}
\newcommand\rhol{\ensuremath{\rho^l}}
\newcommand\rhos{\ensuremath{\rho^s}}
\newcommand\kl{\ensuremath{\kappa^l}}
\newcommand\ks{\ensuremath{\kappa^s}}
\newcommand\hl{\ensuremath{h^l}}
\newcommand\hs{\ensuremath{h^s}}
\newcommand\gl{\ensuremath{g^l}}
\newcommand\gs{\ensuremath{g^s}}
\newcommand\vl{\ensuremath{\vec{v}^l}}
\newcommand\vs{\ensuremath{\vec{v}^s}}

\newcommand\norm[1]{\ensuremath{\lVert #1 \rVert}}

\newcommand*\widefbox[1]{\fbox{\hspace{2em}#1\hspace{2em}}}

% from phd
% Comments
% \usepackage{todonotes}
\usepackage{todonotes}
\newcommand{\comment}[2][]{\todo[backgroundcolor=yellow!50!white, caption={#2}, inline, size=\small, #1]{\renewcommand{\baselinestretch}{0.5}\selectfont#2\par}}

\newcommand\Rref{\ensuremath{\rho_{\text{ref}}}}
\newcommand\Tref{\ensuremath{T_{\text{ref}}}}
\newcommand\Wlref{\ensuremath{w_{\text{ref}}^l}}
\newcommand\rref{\ensuremath{\rho_{\text{ref}}}}
\newcommand\betaT{\ensuremath{\beta_T}}
\newcommand\betaWl{\ensuremath{\beta_{\wl}}}
\newcommand\betaWlC{\ensuremath{\beta_{\avg{w_\text{C}}^l}}}
\newcommand\betaWlCR{\ensuremath{\beta_{\avg{w_\text{Cr}}^l}}}

%====================================================================
%%%% NEWLY ADDED 
\renewcommand\v{\ensuremath{\vec{v}}}  
\newcommand\gravity{\ensuremath{\vec{g}}}
\newcommand\Fv{\ensuremath{F_{\text{v}}}}  
\newcommand\rhog{\ensuremath{\rho \vec{g}}}  
\newcommand\rhobouss{\ensuremath{\widetilde{\rho}}} 
\newcommand\heaviside{\ensuremath{H}}  
\newcommand\levelset{\ensuremath{\alpha(x)}}
\newcommand{\crochet}[1]{ \left[  #1  \right] } 
\newcommand{\mix}[1]{\ensuremath{\widehat{#1}}} % \widehat % \overbrace % \widetilde
\newcommand{\cimlib}{\textsf{CimLib}}
\newcommand\I{\ensuremath{\mat{\text{I}}}}  


\newcommand{\red}[1]{{\color{Red} #1}}
\newcommand{\blue}[1]{{\color{Blue} #1}} 






\title{Levelset-mixing of Navier-Stokes equations with Boussinesq:\\ Incompressible multiphase flow }
\author{Ali SAAD}
\begin{document}
\maketitle
%\dosecttoc
%\faketableofcontents
%\secttoc
%\clearpage
%====================================================================
\section{Introduction}
This report addresses the volume force added by the Boussinesq approximation to the incompressible Navier Stokes equations
when we solve a multiphase flow. Such flow is characterized by the presence of at least two immiscible fluids. \\
The same situation is encountered when treating for instance the SMACS benchmark case before the onset of solidification, where we have two distinct fluids: air and liquid metal.
The flow at this stage is driven by thermal convection using side exchangers: one is cooler than the liquid metal, while the other is hotter.
Since for now solidification is not accounted for, there will no mention of the Darcy interfacial terms. They will be discussed later.
%====================================================================
\section{Monophase flow: momentum conservation}
In the case of a single phase fluid flow, the incompressible Navier Stokes equations, with the only external volume 
force applied to the fluid being its body weight, are written as:
\begin{align}
\label{eq:NS_monophase}
& \rho \brac{\frac{\partial \v}{\partial t} + \brac{\nabmat \v} \v } = \nabvec \cdot \sigma + \rhog \\ 
& \nabla \cdot \v = 0
\end{align}
where $\rho$, $v$, $\sigma$ and $\gravity$ are respectively the fluid's density, fluid's velocity, stress tensor and the gravity vector.
Now, suppose that density is subject to thermal variations, then the Boussinesq holds a good approximation if these variations remain small (order of magnitude given later).
The approximation is introduced by a temperature-dependent density, $\rhobouss$, in the weight force:
\begin{align}
\label{eq:boussinesq}
\rhobouss % &= \rho_0 \theta \\ 
          	&= \rho_0 \brac{1-\betaT \brac{T-\Tref}}
\end{align}
while the inertial term is kept at reference density $\rho_0 = \rhobouss(\Tref)$. This is acceptable as long as the relation
$\betaT \brac{T-\Tref} \ll 1$ is satisfied.
\comment{the only reference that states this fact is FLUENT's reference guide. I should search for more formal source}

%====================================================================
\section{Multiphase flow: momentum conservation}
The monolithic Eulerian appraoches are very interesting in the computational fluid field as they rely on a fixed mesh
where all fluids are treated as a single one. This fluid's thermophysical and mechanical properties are obtained by a 
mixture of those fluids properties. The level set is one of the most suited approaches for multiphase flows.
\subsection{Levelset formulation}
Introducing the levelset function $\alpha$ (signed distance function) helps tracking implicitly the interface between two fluids, $F_1$ and $F_2$.
The distance between any point in space, $x$, and the interface, $\Gamma$, also denoted $d_{\Gamma} \brac{x}$, is used to track the interface as follows:
\begin{align}
\levelset = 
\begin{cases}
 d_{\Gamma} \brac{x} 		& \text{ if } x \in F_1 \\ 
 -d_{\Gamma} \brac{x}	 	& \text{ if } x \in F_2 \\ 
 0 							& \text{ if } x \in \Gamma_{F1, F2} 
\end{cases}
\end{align}
However, the previous function gives only the distance information, thus useless to mix the fluids properties.
The solution is to process the levelset to extract a presence function or the \emph{Heaviside} function, for each fluid.
In the literature, many definitions exist, depending on the function's steepness, continuity or differentiability. In our case, we consider the 
smoothed Heaviside based on a sinusoidal function, given by:
\begin{align}
\heaviside\brac{\levelset} = 
\begin{cases}
	1  & \text{ if } \levelset < \epsilon \\
    0  & \text{ if } \levelset > \epsilon \\  
    \frac{1}{2} \brac{1+ \frac{\levelset}{\epsilon} + \frac{1}{\epsilon}\sin \brac{\frac{\pi \levelset}{\epsilon }} } & \text{ if } - \epsilon \leq \levelset \leq \epsilon
\end{cases}
\end{align}
where the interval $\crochet{-\epsilon; \epsilon}$ is an artificial interface thickness around the zero distance. 
It should be noted that a sinusoidal-based Heaviside is continuously differentiable.
It is clear from the definition above, that the sum of the Heaviside functions for all the fluids is equal to $1$
in any point of the mesh. Bearing in mind that $\heaviside^{F_2} = 1-\heaviside^{F_1}$, the properties $\psi^{F_1}$ and $\psi^{F_2}$ 
are mixed using the Heaviside functions of both fluids:
\begin{align}
\label{eq:mixing}
\mix{\psi} = \heaviside^{F_1} \psi^{F_1} + \heaviside^{F_2} \psi^{F_2}
\end{align}

\subsection{Mixed Navier Stokes equations}
In the case of 2 incompressible fluids, the Navier Stokes system writes: 
\begin{align}
\label{eq:NS_multiphase}
& \mix{\rho} \brac{\frac{\partial \v}{\partial t} + \brac{\nabmat \v} \v } = \nabvec \cdot \mix{\sigma} + \mix{\rho} g \\ 
& \nabla \cdot \v = 0
\end{align}
where the density and the stress tensor are mixed using the previously defined Heaviside function. Considering that the concerned fluids 
are Newtonian, the stress tensor can be expressed in terms of strain rate tensor as: $\sigma = 2\mu \dot{\epsilon}$, hence Navier Stokes 
equations are updated accordingly:
\begin{align}
\label{eq:NS_multiphase_2}
& \mix{\rho} \brac{\frac{\partial \v}{\partial t} + \brac{\nabmat \v} \v } = 2\mix{\mu} \nabvec \cdot \mix{\dot{\epsilon}} + \mix{\rho} g \\ 
& \nabla \cdot \v = 0
\end{align}




\end{document}
