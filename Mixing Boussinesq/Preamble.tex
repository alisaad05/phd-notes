\usepackage[utf8]{inputenc}
\usepackage{amsmath,empheq}
\usepackage{amsfonts}
\usepackage{amssymb}
\usepackage{fourier}
\usepackage[left=2.5cm,right=2.5cm,top=2.5cm,bottom=2.5cm]{geometry}
\author{Ali SAAD}
\title{Formulation of Conservation Equations for the Shrinkage-Induced Surface Deformation}
%====================================================================
%%%   PACKAGES   %%%

%\usepackage{xcolor}
\usepackage[dvipsnames]{xcolor} 
\definecolor{halfgray}{gray}{0.55}
\definecolor{webgreen}{rgb}{0,.5,0}
\definecolor{webbrown}{rgb}{.6,0,0}
\definecolor{Maroon}{cmyk}{0, 0.87, 0.68, 0.32}
\definecolor{RoyalBlue}{cmyk}{1, 0.50, 0, 0}
\definecolor{Black}{cmyk}{0, 0, 0, 0}


\usepackage[pdftex,hyperfootnotes=false,pdfpagelabels]{hyperref}
\hypersetup{%
    %draft,	% = no hyperlinking at all (useful in b/w printouts)
    colorlinks=true, linktocpage=true, pdfstartpage=3, pdfstartview=FitV,%
    % uncomment the following line if you want to have black links (e.g., for printing)
    %colorlinks=false, linktocpage=false, pdfborder={0 0 0}, pdfstartpage=3, pdfstartview=FitV,% 
    breaklinks=true, pdfpagemode=UseNone, pageanchor=true, pdfpagemode=UseOutlines,%
    plainpages=false, bookmarksnumbered, bookmarksopen=true, bookmarksopenlevel=1,%
    hypertexnames=true, pdfhighlight=/O,%nesting=true,%frenchlinks,%
    urlcolor=webbrown, linkcolor=RoyalBlue, citecolor=webgreen, %pagecolor=RoyalBlue,%
    %urlcolor=Black, linkcolor=Black, citecolor=Black, %pagecolor=Black,%
} 

\usepackage{cancel}
\usepackage{import}

\usepackage{tabularx} % better tables
	\setlength{\extrarowheight}{3pt} % increase table row height
\newcommand{\tableheadline}[1]{\multicolumn{1}{c}{\spacedlowsmallcaps{#1}}}
\newcommand{\myfloatalign}{\centering} % to be used with each float for alignment
\usepackage{caption}
\captionsetup{format=hang,font=small}
%\usepackage{subfig}
\usepackage{float}
\usepackage{subcaption}

%\usepackage{minitoc}
%====================================================================
%%%   COMMANDS   %%%
%\renewcommand{matrix}[1]{#1 \rangle} 
\newcommand{\avg}[1]{\left\langle #1 \right\rangle}
\newcommand{\brac}[1]{ \left(  #1  \right) } 
\newcommand{\mtrx}[1]{{\underline{\underline{#1}}} }
\newcommand\mat[1]{\ensuremath{\underline{#1}}} 

\newcommand\nabvec{\ensuremath{\vec{\nabla}}}
\newcommand\nabmat{\ensuremath{\mat{\nabla}}}
\newcommand\temp[1]{\ensuremath{\frac{\partial}{\partial t} \brac{#1} }}
\newcommand\tempn[1]{\ensuremath{\frac{\partial #1}{\partial t} }}  
 
\newcommand\vit{\ensuremath{\avg{\vec{v}^l}}}  
\newcommand\strainrate{\ensuremath{\avg{{\dot{\mat{\varepsilon}}}^l}}}

\newcommand\wavg{\ensuremath{\avg{w}}}
\newcommand\wl{\ensuremath{\avg{w}^l}}
\newcommand\ws{\ensuremath{\avg{w}^s}}
\newcommand\rhol{\ensuremath{\rho^l}}
\newcommand\rhos{\ensuremath{\rho^s}}
\newcommand\kl{\ensuremath{\kappa^l}}
\newcommand\ks{\ensuremath{\kappa^s}}
\newcommand\hl{\ensuremath{h^l}}
\newcommand\hs{\ensuremath{h^s}}
\newcommand\gl{\ensuremath{g^l}}
\newcommand\gs{\ensuremath{g^s}}
\newcommand\vl{\ensuremath{\vec{v}^l}}
\newcommand\vs{\ensuremath{\vec{v}^s}}

\newcommand\norm[1]{\ensuremath{\lVert #1 \rVert}}

\newcommand*\widefbox[1]{\fbox{\hspace{2em}#1\hspace{2em}}}

% from phd
% Comments
% \usepackage{todonotes}
\usepackage{todonotes}
\newcommand{\comment}[2][]{\todo[backgroundcolor=yellow!50!white, caption={#2}, inline, size=\small, #1]{\renewcommand{\baselinestretch}{0.5}\selectfont#2\par}}

\newcommand\Rref{\ensuremath{\rho_{\text{ref}}}}
\newcommand\Tref{\ensuremath{T_{\text{ref}}}}
\newcommand\Wlref{\ensuremath{w_{\text{ref}}^l}}
\newcommand\rref{\ensuremath{\rho_{\text{ref}}}}
\newcommand\betaT{\ensuremath{\beta_T}}
\newcommand\betaWl{\ensuremath{\beta_{\wl}}}
\newcommand\betaWlC{\ensuremath{\beta_{\avg{w_\text{C}}^l}}}
\newcommand\betaWlCR{\ensuremath{\beta_{\avg{w_\text{Cr}}^l}}}

%====================================================================
%%%% NEWLY ADDED 
\renewcommand\v{\ensuremath{\vec{v}}}  
\newcommand\gravity{\ensuremath{\vec{g}}}
\newcommand\Fv{\ensuremath{F_{\text{v}}}}  
\newcommand\rhog{\ensuremath{\rho \vec{g}}}  
\newcommand\rhobouss{\ensuremath{\widetilde{\rho}}} 
\newcommand\heaviside{\ensuremath{H}}  
\newcommand\levelset{\ensuremath{\alpha(x)}}
\newcommand{\crochet}[1]{ \left[  #1  \right] } 
\newcommand{\mix}[1]{\ensuremath{\widehat{#1}}} % \widehat % \overbrace % \widetilde





